\documentclass[12pt,a4paper,twoside]{article}
\usepackage[utf8]{inputenc}
\usepackage{amsmath,amsfonts,amssymb}
\usepackage{graphicx}
\usepackage{natbib}
\usepackage{siunitx}
\usepackage[top=1cm,left=2.5cm,right=2.5cm]{geometry}
\usepackage[pdfborder=0]{hyperref}

\setlength{\parindent}{0pt}
\setlength{\parskip}{15pt}

\pagestyle{empty}

\renewcommand{\baselinestretch}{1.15}
\renewcommand{\labelenumi}{{\bf \arabic{enumi}.}}
\newcommand{\abs}[1]{\ensuremath{\left \vert #1 \right \vert}}

\input{journals.tex}


\begin{document}

\hrule

\textbf{Diffusion in a homogeneous, plane-parallel medium}

  The goal of this exercise is to determine the dispersion of a sharply peaked
  pulse of radiative energy with Monte Carlo techniques\footnote{See
    \citet{Noebauer2012} and \citet{Abdikamalov2012} for more details on this
  problem.}.  In particular, we consider a pulse of radiation with energy
  $\mathcal{E}_{\mathrm{tot}}$, which follows a $\delta$ distribution. It
  resides in a plane-parallel box, extending form $-L/2$ to $L/2$. The grey
  scattering opacity $\chi$ is constant and chosen such that the total optical
  depth along the symmetry axis amounts to $\tau_{\mathrm{tot}} = 10^2$. Thus,
  the optical depth of the considered system is high enough for the diffusion
  approximation to be applicable and the analytic solution for the evolution of
  the pulse is given by
  \begin{equation}
    E(x,t) = \mathcal{E}_{\mathrm{tot}} \frac{1}{\sqrt{4\pi D t}} \exp\left( -\frac{x^2}{4
    D t},
    \right)
    \label{eq:solution}
  \end{equation}
  with the diffusion constant\footnote{Here, $c$ denotes the speed of
  light.} 
  \begin{equation}
    D = \frac{c}{3 \chi}.
    \label{eq:diffusion_constant}
  \end{equation}
  Note the difference between energy $\mathcal{E}$ and energy density $E$:
  $[\mathcal{E}] = \mathrm{erg}$, $[E] = \mathrm{erg\,cm^{-1}}$.

  \begin{enumerate}
    \item Write a program to determine the evolution of the energy pulse with
      Monte Carlo techniques. Use the parameters $\mathcal{E}_{\mathrm{tot}} =
      \SI[retain-unity-mantissa=false]{1e8}{erg}$, $\chi =
      \SI[retain-unity-mantissa=false]{1e2}{cm^{-1}}$ and $L
      = \SI{1}{cm}$
    \item Compare the Monte Carlo results with the analytic solution
      (\ref{eq:solution}) at $t = 5
      \times 10^{-12}$, $10^{-11}$, $2 \times 10^{-11}$ and $5 \times 10^{-11} \,
      \mathrm{s}$:
  \end{enumerate}

  The following hints may help you:
  \begin{itemize}
    \item distribute the initial radiation energy evenly onto the Monte
      Carlo packets
    \item initialise all packets with isotropic directions at $x = 0$
    \item distances to the next interaction location follow as usual from
      $\rho(l) = \chi \exp(-\chi l)$
    \item assume that all interactions occur as resonant scatterings, i.e.\ the
      packet energy remains constant and the emergent directions are isotropic
    \item in order to determine the distribution of radiative energy at time
      $t$, you have to track the total pathlength $l_{\mathrm{tot}}$ packets cover
    \item you follow the propagation of a particular packet until
      $l_{\mathrm{tot}}$ is equal to the maximum length packets can travel, $c
      \Delta t$
    \item the extent of the box and the time snapshots have been chosen such
      that you don't have to worry about packets crossing the ``boundaries'' of
      the box
    \item to illustrate the radiation field energy density at time $t$, make a
      histogram of the packet location and multiply it with the packet energy.
      If you use \texttt{python} and \texttt{matplotlib}, have a look at the
      routine \texttt{matplotlib.pylab.hist} and in particular at its keyword
      argument \texttt{weights}.
    \item your solution should looks similar to Figure \ref{fig:sol}
  \end{itemize}

  \begin{figure}[h]
    \centering
    \includegraphics[width=\textwidth]{diffusion_results}
    \caption{Comparison between the Monte Carlo simulation results and the
      analytic solution (\ref{eq:solution}) for the requested snapshots ($10^4$
    packets were used).}
    \label{fig:sol}
  \end{figure}

\bibliography{astrouli.bib}
\bibliographystyle{aa}


  
\end{document}
