\documentclass[12pt,a4paper,twoside]{article}
\usepackage[utf8]{inputenc}
\usepackage{amsmath,amsfonts,amssymb}
\usepackage{graphicx}
\usepackage[top=1cm,left=2.5cm,right=2.5cm]{geometry}
\usepackage[pdfborder=0]{hyperref}

\setlength{\parindent}{0pt}
\setlength{\parskip}{15pt}

\pagestyle{empty}

\renewcommand{\baselinestretch}{1.15}
\renewcommand{\labelenumi}{{\bf \arabic{enumi}.}}
\newcommand{\abs}[1]{\ensuremath{\left \vert #1 \right \vert}}


\begin{document}

\hrule

\begin{enumerate}
\item \textbf{Escape Probability from a Homogeneous Slab}

  Before turning to the problem of radiation escaping from a homogeneous sphere,
  as described in the handout, we consider an infinitely extended slab with
  constant emissivity and absorption opacity. Again, we aim at determining the
  probability for photons to escape from this object by using a Monte Carlo
  approach. The same basic principles as discussed in the handout apply here,
  but a simpler geometrical configuration is encountered.

  We approach the problem in the following way:
  \begin{itemize}
    \item we assume that the slab extents indefinitely along the $y$ and $z$
      directions. Along the $x$-axis, it reaches from $0$ to $L$.
    \item in this plane-parallel situation, we only have to account for the
      $x$-dependence of all quantities. Directions relative to this symmetry
      axis are measured by the angle $\theta$. Instead of using this angle
      directly, however, it is commonly more convenient to work with its
      cosine $\mu = \cos \theta$ in radiative transfer applications.
    \item a large number of Monte Carlo ``packets'' (i.e.\ photons) representing
      the radiation field is introduced. They are uniformly distributed within
      the slab of width $L$. 
    \item since the radiation field is isotropic, the initial directions of the
      photons follow from the probability distribution $\rho(\mu) =
      \frac{1}{2}$
    \item the Monte Carlo packets will undergo an interaction with the
      probability $\rho(\tau) = \exp(-\tau)$. Here, the optical depth is
      introduced which is the product of the absorption opacity $\chi$ and the
      pathlength the packet/photon covers: $\tau = \chi s$\footnote{This
        relation is only valid as long as the opacity is constant. Otherwise an
      integral over the pathlength has to be used.}. Note that $s$
      refers to the actual distance packets/photons travel, not just its
      projection onto the symmetry axis.
    \item for each Monte Carlo packet, the distance $d$ to the surface of the
      slab has to be determined. Note that in the plane-parallel case, packets
      may escape from either side of the slab. This distance has to be converted
      into an optical depth, $\tau_{\mathrm{edge}}$.
    \item finally, the number of escaping packets may be calculated: packets
      which fulfil $\tau > \tau_{\mathrm{edge}}$ reach the surface of the slab
      before interacting and escape.
  \end{itemize}
  \pagebreak

  Following this basic outline, your tasks include:
  \begin{enumerate}
    \item calculate the escape probability slabs of different optical thickness,
      i.e.\ for different values of $\tau_{\mathrm{sl}} = \chi L$, using a reasonable number of
      Monte Carlo packets.
    \item estimate the error of your Monte Carlo results. Use the basic property
      of Monte Carlo techniques here, namely that their accuracy increases with
      $N^{-1/2}$.
    \item compare your results with the analytic solution
      $P_{\mathrm{esc}}(\tau_{\mathrm{sl}})$:
      \begin{equation}
      P_{\mathrm{esc}}(\tau) = \frac{1}{2 \tau} \left[ 1 +
        \exp(-\tau) \left(\tau - 1 + \tau^2 \exp(\tau) \mathrm{Ei}(-\tau) \right)
      \right] 
      \end{equation}
      Do your results agree within their respective uncertainties?
  \end{enumerate}

  \textit{Note:} In the analytic solution the so-called ``exponential integral''
  appears:
  \begin{equation*}
    \mathrm{Ei}(x) = \int_{-\infty}^x \mathrm{d}t \frac{\exp{t}}{t}
  \end{equation*}
  Have a look at \texttt{scipy.special.expi} if you want to use this expression
  within your \textsc{Python} program. 
\end{enumerate}

\end{document}
