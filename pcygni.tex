\documentclass[12pt,a4paper,twoside]{article}
\usepackage[utf8]{inputenc}
\usepackage{amsmath,amsfonts,amssymb}
\usepackage{graphicx}
\usepackage{siunitx}
\usepackage[top=1cm,left=2.5cm,right=2.5cm]{geometry}
\usepackage[pdfborder=0]{hyperref}

\setlength{\parindent}{0pt}
\setlength{\parskip}{15pt}

\pagestyle{empty}

\renewcommand{\baselinestretch}{1.15}
\renewcommand{\labelenumi}{{\bf \arabic{enumi}.}}
\newcommand{\abs}[1]{\ensuremath{\left \vert #1 \right \vert}}


\begin{document}

\hrule

\begin{enumerate}
\item \textbf{Sobolev Approximation and P-Cygni Line Profiles}

  In this exercise, the Sobolev Approximation is used to calculate a P-Cygni
  within a Monte Carlo radiative transfer simulation. For this purpose, we
  consider a shell in homologous expansion which is illuminated by a radiative
  source at its lower boundary. Within the expanding material, photons may
  perform resonant line-interactions with one atomic transition only. The
  emergent radiation field is recorded and thus line profile, shaped by the
  interactions within the shell, calculated. Your task is to write a simple
  Monte Carlo program, which solves the propagation of the radiation field. Use
  the Sobolev approximation to treat the line interactions and thus determine
  the P-Cygni line profile in the emergent spectrum. For this problem you may
  adopt the following assumptions:

  \begin{itemize}
    \item the shell extends from $R_{\mathrm{min}} = \SI{1e11}{cm}$ to
      $R_{\mathrm{max}} = \SI{1e13}{cm}$. In velocity space, it covers the
      linear range from $0.0001\,c$ to $0.01 \,c$. 
    \item the shell is entirely composed of hydrogen and the only allowed atomic
      transition is the Lyman-$\alpha$ line at the rest frame wavelength of
      $1215.6\,\mathrm{\AA}$
    \item the material within the shell has constant density and temperature.
      These material properties are such that the Sobolev optical depth of the
      relevant transition amounts to $\tau_{\mathrm{sob}} = 0.1$
    \item at the lower boundary, an isotropic radiative source is located, which
      emits uniformly within the wavelength range $\lambda_{\mathrm{min}} = 1100
      \, \mathrm{\AA}$ and $\lambda_{\mathrm{max}} = 1300 \, \mathrm{\AA}$
  \end{itemize}

  In developing the Monte Carlo routine you may adhere to the following
  strategy:
  \begin{enumerate}
    \item initialise the Monte Carlo packets at the lower boundary. Assign an
      initial frequency and direction (assuming no limb darkening, i.e.\ $N(\mu)
      \mathrm{d} \mu \propto \mu \mathrm{d}\mu$)
    \item packets may only interact at the Sobolev point, i.e.\ at the location
      where their CMF frequency is equal to the rest-frame frequency of the line
      transition. Determine this point using the Doppler effect formula, $\nu_0 =
      \gamma \nu (1 - \beta \mu)$\footnote{For the problem at hand, where $v \ll
      c$, you may set $\gamma = 1$.}
    \item at the Sobolev point, packets interact if $\tau >
      \tau_{\mathrm{sob}}$. The optical depth of the packet is determined in the
      usual way, i.e.\ by $\tau = - \ln z$
    \item if a line interaction occurs, change the packet properties by assuming
      isotropy of the resonant scatterings and by according for 
      energy conservation in the CMF 
    \item follow the propagation of each packet in the described fashion until
      it leaves through either of the two boundaries. Discard the packet if it
      intersects the lower boundary, but record its LF frequency if it emerges
      through the outer boundary
    \item construct a spectrum from the recorded frequencies
  \end{enumerate}

\end{enumerate}
\end{document}
